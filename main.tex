\documentclass{article}
\usepackage[utf8]{inputenc}
\usepackage[english, russian]{babel}
\usepackage{hyperref}
\usepackage{url}
\usepackage{booktabs}
\usepackage{amsfonts}
\usepackage{nicefrac}
\usepackage{microtype}
\usepackage{graphicx}
\usepackage{natbib}
\usepackage[a4paper, margin=1in]{geometry}  % This line sets the margins

\title{\textbf{Резюме}}
\author{\texttt{Фирсов Сергей Андреевич}}
\date{}

\begin{document}

\maketitle


\texttt{4-й курс МФТИ, ФПМИ ПМФ ПМ (ФУПМ), кафедра Интеллектуальный Анализ Данных (от ВЦ РАН)}

\section*{Контактная информация}
\begin{itemize}
    \item \href{https://github.com/Schaft-s}{GitHub}
    \item Телефон: +7 910 248 28 48
    \item Почта: firsov.sa@phystech.edu, seregei.fix@yandex.ru, firsov.sa20@gmail.com
\end{itemize}

\section*{О себе}
Прилежный студент 4-го курса МФТИ, который получил сильную базу знаний от ВУЗа и множества доп организаций, желающий применять все эти знания на практике. Сфера интересов - Машинное обучение. Изучаю различную информацию в этой сфере, слежу за передовыми исследованиями, прохожу все доступные курсы в МФТИ и вокруг него. Кафедра ИАД преподаёт теоретический ML, благодаря обучению там написал  \href{https://github.com/intsystems/2024-Project-142}{(статью)}. Также пишу диплом по ML под руководством Бахтеева Олега Юрьевича, изучаемая тема "Выбор архитектуры модели глубокого обучения с переменным контролем сложности".

\section*{Проекты}
\begin{itemize}
    \item \textbf{Статья "Классификация ОКПД2 кодов"}: Работа продолжается, готовится к публикации. \href{https://github.com/intsystems/2024-Project-142}{Проект на GitHub}.
    \item Проект со смены в Сириусе: аппроксимация тензоров \(E\) и \(\mu\) цифрового керна породы методами численного усреднения и с применением нейросетей.
    \item Частный проект по распознаванию среза дерева на фото.
    \item Пет-проект по распознаванию монет на фотографии.
    \item Базовые проекты для знакомства со стандартными датасетами и алгоритмами: NN и CNN на mnist, SVM с нуля, дообучение VLM для VQA, решение задачи NER 3-мя способами, decision tree на iris, НБК, кредитный скоринг...
\end{itemize}

\section*{Небольшая выдержка из дополнительного образования в сфере ML}
\begin{itemize}
    \item Продвинутые методы машинного обучения от ЦК МФТИ, диплом о профессиональной переподготовке*
    \item Смена в Сириусе "Матричные методы и искусственный интеллект". Диплом о проф подготовке* 
    \item Курс Воронцова по машинному обучению. С семинарами Грабового.
    \item Учебник ШАДа по машинному обучению.
    \item 2 курса по машинному обучению от Нейчева.
    \item Учебник по NLP от Елены Войта. (в процессе)
    \item Курс по DL от кафедры ИАД. (в процессе)  
    
* отдадут после получения диплома о высшем образовании.
\end{itemize}

\section*{Опыт работы:}
\begin{itemize}
\item  Семинарист по С++ в МФТИ.
\item  Репетитор по Python и его применению в анализе данных для студентов ВШЭ.
\item  Большие проекты: из Сириуса и два по компютерному зрению.
\item  Множество пет проектов по ML описанные выше.
\end{itemize}

\section*{IT Навыки:}
\begin{itemize}
\item \textbf{ML}: Активно изучаю, слежу за передовыми исследованиями. Изучил множество дополнительных ресурсов.
\item \textbf{Python} : Использую в проектах, в том числе связанных с машинным обучением. Репетитор по Python для студентов некоторых направлений Высшей школы экономики.
\item \textbf{C++} : Глубокие теоретические знания, преподаю в МФТИ базовый курс, использовал в учебных курсах. (Тут же чистый С)
\item \textbf{SQL} : Знание теории, практическое применение на LeetCode, сдал курс на 10.
\item Также есть знания пo Go, assembler и многому другому.
\end{itemize}



Если Вы читаете это - большое спасибо за уделённое время! 

С удовольствием пройду собеседование для подтверждения моих навыков и готов приступить к работе.

\end{document}
