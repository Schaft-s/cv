\documentclass[a4paper,12pt]{article}
\usepackage[T2A]{fontenc}
\usepackage[utf8]{inputenc}  % для правильного распознавания UTF-8 символов
\usepackage[russian]{babel}  % для поддержки русского языка
\usepackage{hyperref}
\usepackage{amsmath}  % Для математических символов, если они будут использоваться
\usepackage[a4paper, left=2cm, right=2cm, top=2cm, bottom=2cm]{geometry}

\title{\textbf{Резюме}}
\author{\texttt{Фирсов Сергей Андреевич}}
\date{}

\begin{document}

\maketitle

\texttt{4-й курс МФТИ, ФПМИ, кафедра Интеллектуальный Анализ Данных (от ВЦ РАН)}

\section*{Контактная информация}
\begin{itemize}
    \item \href{https://github.com/Schaft-s}{GitHub}
    \item Телефон: +7 910 248 28 48
    \item Почта: firsov.sa@phystech.edu, seregei.fix@yandex.ru, firsov.sa20@gmail.com
\end{itemize}

\section*{О себе}
Студент выпуского курса бакалавриата МФТИ, моя сфера интересов - Машинное обучение, в частности Computer Vision и NLP. Изучаю различную информацию в этой сфере, слежу за передовыми исследованиями, прохожу все доступные курсы в МФТИ и вокруг него. Кафедра ИАД преподаёт теоретический ML, благодаря обучению там написал  \href{https://github.com/intsystems/2024-Project-142}{(статью)}. Также пишу диплом по ML под руководством Бахтеева Олега Юрьевича, изучаемая тема "Выбор архитектуры модели глубокого обучения с переменным контролем сложности". Кроме того занимаюсь различными исследованиями в смежных областях, один из примеров - применение теории информации к анализу эффектов обучения нейросетей.

\section*{Опыт работы:}
\begin{itemize}
\item  Проектный офис ФПМИ (Junior, 6 месяцев) - решение задач детекции объектов, для дальнейшего управления ими на АСУ ТП. (от известной нефтяной компании)
\item  Работа над проектом классификации ОКПД 2 кодов, на основе которой написана "статья" для курса от кафедры ИАД. 
\item  Хакатон от ЦК МФТИ - "тим лид" команды. (ниже описана задача)
\item  Семинарист по С++ в МФТИ.
\item  Репетитор по Python и его применению в анализе данных для студентов ВШЭ.
\item  Большие проекты: Кейс (решение задачи e-commerce), кейс-проект из Сириуса и два по компьютерному зрению.
\end{itemize}

\section*{Проекты}
\begin{itemize}
    \item Хакатон от МФТИ, решение задачи от ВК, ранжирование ответов на вопрос. (NLP би и кросс-энкодеры)
    \item Кейс (решения задач e-commerce) по сегментации изображения и генерации нового фона + подписи.
    \item Кейс со смены в Сириусе: аппроксимация тензоров \(E\) и \(\mu\) цифрового керна породы методами численного усреднения и с применением нейросетей.
    \item \textbf{Статья "Классификация ОКПД2 кодов"}: не выпущена, работа приостановлена. \href{https://github.com/intsystems/2024-Project-142}{Проект на GitHub}.
    \item Частный проект по распознаванию среза дерева на фото. Пет-проект по распознаванию монет на фотографии.
    \item Исследование (пока не статья) по применению Теории Информации для анализа нейросетевых эффектов - гроккинг, double descent.
\end{itemize}

\section*{Небольшая выдержка из дополнительного образования в сфере ML}
\begin{itemize}
    \item Продвинутые методы машинного обучения от ЦК МФТИ, диплом о профессиональной подготовке*. 
    \item Смена в Сириусе "Матричные методы и искусственный интеллект". Диплом о проф подготовке* 
    \item Курс Воронцова + 2 курса (basic and advanced) от Нейчева по машинному обучению, доп курсы по DL и Time Series от кафедры ИАД.
    \item Учебник ШАДа по машинному обучению + учебник по NLP от Елены Войта. 
    
* отдадут после получения диплома о высшем образовании.

\end{itemize}


\section*{IT Навыки:}
\begin{itemize}
\item \textbf{ML}: Активно изучаю, слежу за передовыми исследованиями. Изучил множество дополнительных ресурсов.
\item \textbf{Python} : Использую в проектах, в том числе связанных с машинным обучением. Репетитор по Python для студентов некоторых направлений Высшей школы экономики.
\item \textbf{C++} : Глубокие теоретические знания, преподаю в МФТИ базовый курс, использовал в учебных курсах. (Тут же чистый С)
\item \textbf{SQL} : Знание теории, сдал курс на 10. На практике давно не пользовался.
\item Также есть знания пo Go, assembler и многому другому.
\end{itemize}

Если Вы читаете это - большое спасибо за уделённое время! 

С удовольствием пройду собеседование для подтверждения моих навыков и готов приступить к работе.

\end{document}
